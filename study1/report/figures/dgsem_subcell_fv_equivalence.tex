% !TeX root = ../main.tex
\documentclass{standalone}
\usepackage{tikz}
\usetikzlibrary{arrows,calc}
\begin{document}
	\begin{tikzpicture}[x=1cm, y=1cm]
		
		\pgfmathsetmacro\linewd{\linewidth/1cm} % the linewidth
		\pgfmathsetmacro\cl{0.75*\linewd} % characteristic length
		\pgfmathsetmacro\w{0.075*\cl} % height/depth of finite volume
		
		\tikzstyle{element line}=[thick];
		\tikzstyle{fv line}=[thick, black, <->, >=stealth];
		\tikzstyle{fv label}=[black, fill=white, inner sep=2pt];
		\tikzstyle{quadrature point}=[circle, inner sep=2pt, fill=red];
		\tikzstyle{flux point}=[cross out, inner sep=3pt, draw=blue];
		
		% quadrature and flux points pre-calculated for N=3
		\def\quadpoints{{0, 0.276393, 0.723607, 1}};
		\def\fluxpoints{{0, 0.0833333, 0.5, 0.916667, 1}};
		
        \draw[element line] (0,0) -- (\cl,0);
        
        \foreach \i in {0, 1, ..., 3}{
        	\node[quadrature point] at (\quadpoints[\i]*\cl,0) {};
        	\node at (\quadpoints[\i]*\cl,0) [above=2pt] {{\color{red}$\vect{\eulerref{F}}_\i$}};
        	\draw[fv line] (\fluxpoints[\i]*\cl,-\w) -- node [fv label] {$w_\i$} (\fluxpoints[\i+1]*\cl,-\w);
    	}
    	
    	\foreach \i in {0, 1, ..., 4}{
    		\node[flux point] at (\fluxpoints[\i]*\cl,0) {};
    		\node at (\fluxpoints[\i]*\cl,0) [below=2pt] {{\color{blue}$\hat{\vect{\eulerref{F}}}_\i$}};
    		\draw[element line] (\fluxpoints[\i]*\cl,-1.15*\w) -- (\fluxpoints[\i]*\cl,-0.85*\w);
    	}
        
	\end{tikzpicture}
\end{document}
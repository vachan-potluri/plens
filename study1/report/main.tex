% !TeX spellcheck = en_GB
\documentclass[a4paper,11pt,oneside]{article}

\usepackage[vmargin=0.8in,outer=0.65in,inner=0.8in]{geometry}

\usepackage{newtxtext} % for roman style text and math
\usepackage{newtxmath}

\usepackage{etoolbox} % for defining and using boolean vars
\usepackage{setspace} % for changing line spacing. Using \linespread is not recommended
\usepackage{xspace} % for space after text macros
\usepackage{enumitem} % for changing settings of enumerate and itemize lists
\usepackage{caption,subcaption} % for changing caption settings
\usepackage{graphicx} % to include figures
\usepackage{float} % for modifications to figure position
\usepackage{standalone} % for stanalone tikz pictures
\usepackage{tikz}
\usetikzlibrary{arrows,calc,shapes,positioning,trees,matrix}
\usepackage{amsmath} % for math support
\usepackage{mathtools} % for math support
%\usepackage{upgreek}
\usepackage{accents}
\usepackage{empheq} % for emphasis (boxing etc) around multiple equations
\usepackage{gensymb} % for degree
\usepackage{bm} % for bold math symbols
\usepackage{cancel}
\usepackage{chemformula} % for chemicals
\usepackage{siunitx} % for units
\usepackage{booktabs} % for well spaced tables
%\usepackage{fancyhdr} % for header and footer
\usepackage{epstopdf} % for inserting .eps files as images
\usepackage[
backend=biber,
style=nature,% important, use this style only
citestyle=numeric-comp,
maxbibnames=50,
sorting=none]%important to specify this
{biblatex}
\usepackage{hyperref}
\hypersetup
{
	colorlinks=true,
	linkcolor=black,
	filecolor=black,      
	urlcolor=black,
	citecolor=black,
	pdftitle={APS-2 report},
}

\usepackage[compress,capitalize]{cleveref} % to be loaded after hyperref
% ``capitalize'' to use capital starting letter in reference
% ``compress'' to prevent sorting multiple references and use compressed referencing
\crefname{subsection}{Subsection}{Subsections} % to refer subsections, label them using \label[subsection]{name}
% Assuming that cleveref is loaded using capitalize option

\setlength{\parindent}{2em}
\setlength{\parskip}{0.5em}

\setlist[enumerate]{
	% vertical spacing
	topsep=0em,
	itemsep=0em,
	% horizontal spacing
	labelindent=0em,
	leftmargin=\parindent,
	labelsep=*,
}
\setlist[itemize]{
	% vertical spacing
	topsep=0em,
	itemsep=0em,
	% horizontal spacing
	labelindent=0em,
	leftmargin=\parindent,
	labelsep=*,
}
\captionsetup[figure]{labelfont={sf,bf},textfont={sf}}
\captionsetup[table]{labelfont={sf,bf},textfont={sf}}

%\addbibresource{references-articles.bib}
%\addbibresource{references-books.bib}
%\addbibresource{references-others.bib}

\newcommand{\newword}[1]{\textbf{#1}} % style of text for introducing a new important word or definition
\newcommand{\citear}[1]{\citeauthor{#1} \cite{#1}} % cite author and ref
\newcommand{\ndnum}[1]{\textrm{#1}} % for non-dimensional numbers
\newcommand{\vect}[1]{\ensuremath{\boldsymbol{\mathbf{#1}}}} % for 1st order tensors
\newcommand{\tens}[1]{\underline{\vect{#1}}} % for 2nd order tensors
\newcommand{\pder}[2]{\frac{\partial #1}{\partial #2}} % for partial derivative
\newcommand{\der}[2]{\frac{\sdd #1}{\sdd #2}}
\newcommand{\pprime}{{\prime\hspace{-1pt}\prime}} % double prime (only for superscripts of normal size math text. will not work for any other levels like superscript of super/subscript)
\newcommand{\ppprime}{{\prime\hspace{-1pt}\prime\hspace{-1pt}\prime}} % triple prime (only for superscripts of normal size math text. will not work for any other levels like superscript of super/subscript)
\newcommand{\norm}[1]{\left\lVert#1\right\rVert}
\newcommand{\comment}[1]{
	{\sffamily\textcolor{red}{#1}}%
} % for comments
\newcommand{\outline}[1]{
	\iftoggle{editing}{
		\noindent\rule{\linewidth}{2pt}
		\textcolor{blue!60!black}{{#1}}
		\vspace*{-1em}\noindent\rule{\linewidth}{2pt}}{
		%
	}
} % for writing outlines to sections. shows up only when 'editing' is true
\newcommand{\pens}{\textsf{pens2D}\xspace} % shortcut for pens2D
\newcommand{\plens}{\textsf{PLENS}\xspace} % shortcut for PLENS
\newcommand{\sdi}{\ensuremath{\,\text{d}}} % small 'd' for integration
\newcommand{\sdd}{\ensuremath{\text{d}}} % small 'd' for differentiation
\newcommand{\abs}[1]{\ensuremath{\left\lvert#1\right\rvert}} % absolute values
\DeclareMathOperator{\diag}{diag} % for diagonal matrices
\DeclareMathOperator{\sgn}{sgn} % sign
\newcommand{\res}{\ensuremath{\mathcal{L}}} % for residual (describing RK method)
\newcommand{\bigoh}{\ensuremath{O}} % 'order of' symbol
\newcommand{\dealii}{\textsf{deal.II}\xspace}
\newcommand{\pv}{\textsf{ParaView}\xspace}
\newcommand{\apsi}{APS-1\xspace}
\newcommand{\defeq}{\ensuremath{:=}} % defined to be equal to
\newcommand{\refelemone}{\ensuremath{\mathcal{R}}} % reference element in 1d
\newcommand{\elemmapone}{\ensuremath{\mathcal{M}}} % element mapping in 1d
\newcommand{\linalgmat}[1]{\ensuremath{\undertilde{#1}}} % for linear algebra matrix
\newcommand{\linalgvect}[1]{\ensuremath{\underline{#1}}} % linear algebra vector
\newcommand{\textdg}{\text{DG}} % text DG
\newcommand{\textfv}{\text{FV}} % text FV
\newcommand{\avg}[1]{\ensuremath{\left\{ \mskip-5mu \left\{ #1 \right\} \mskip-5mu \right\}}} % for average
\newcommand{\textch}{\text{Ch}} % abbrevation for Chandrashekhar
\newcommand{\textln}{\text{ln}} % for superscript ln in chandrashekhar flux
\newcommand{\textho}{\text{HO}} % abbrevation for high order
\newcommand{\textlo}{\text{LO}} % abbrevation for low order
\newcommand{\trouble}{\ensuremath{\mathbb{E}}} % for energy in higher modes
\newcommand{\threshold}{\ensuremath{\mathbb{T}}} % threshold for Persson's detector
\newcommand{\textmax}{\text{max}} % max in text
\newcommand{\textmin}{\text{min}} % min in text
\newcommand{\textsh}{\text{sh}} % subscript for shock impingement location in Degrez case
\newcommand{\eulerphy}[1]{\ensuremath{\mathcal{#1}}} % euler equation variables in physical space
\newcommand{\eulerref}[1]{\ensuremath{#1}} % euler equation variables in physical space

% Define a boolean 'editing' to have 'outline' display
\providetoggle{editing}

\addbibresource{references.bib}

\begin{document}
\onehalfspacing % line spacing set to 1.5
\raggedbottom % allow for text height to vary between pages

\tableofcontents

\section{Introduction}
\label{sec:intro}

Recently, lot of efforts are being made to bring in high order numerical methods for main-stream CFD simulations [Wang et al 2013]. One of the promising methods in this collection is the DG method. The use of DG method for nonlinear hyperbolic problems was propelled to fore-front by the work of \citear{cockburnShu1998a}. When combined with the numerical fluxes, studied in quite detail for FVM, the element-local nature of these methods makes them favourable for complex geometries and HPC \cite{cockburnShu2001}. However, they also require additional mechanisms for shock capturing and this has been the main line of work for these methods.

For a given mesh, the DG solution takes significantly higher computing time depending on the polynomial degree used. But it also gives better accuracy and resolution provided a `good' limiter is used. Instead of comparing the solutions on a given mesh, it is more useful to compare the accuracy level obtained for a given computational cost [Wang et al 2013]. In such a comparison, higher order methods perform well on cases dominated by smooth regions [refs]. However, the same comparison gives inconclusive results for flows dominated by strong discontinuities [refs].

In this work, we compare the accuracy and computational cost for 4 different polynomial interpolation values. In doing this, we keep the degrees of freedom constant which results in a constant memory load on the CPU. We aim to study and conclude what type of cases have advantageous high order solutions. We also aim to answer to what extent will increasing polynomial order significantly effect the accuracy of the solution, and when does the computational cost offset any benefit in this regard.



\section{Governing equations}
\label{sec:gov-eq}

This work simulates the 3d Euler equations given by
\begin{equation*}
	\pder{\vect{\eulerphy{U}}}{t} + \sum_{i=1}^{3} \pder{\vect{\eulerphy{F}}_i}{x_i} = \vect{0},
	\label{eq:3d_euler}
\end{equation*}
where $\vect{\eulerphy{U}} \defeq \left[ \rho\ \rho u_1\ \rho u_2\ \rho u_3\ \rho E \right]^T$ is the state vector, and the flux vectors are given by
\begin{equation*}
	\vect{\eulerphy{F}}_i(\vect{\eulerphy{U}}) \defeq
	\begin{bmatrix}
		\rho u_i\\
		\rho u_1 u_i\\
		\rho u_2 u_i\\
		\rho u_3 u_i\\
		\rho E u_i
	\end{bmatrix}
	+ p
	\begin{bmatrix}
		0\\
		\delta_{1i}\\
		\delta_{2i}\\
		\delta_{3i}\\
		u_i
	\end{bmatrix}
	\label{eq:3d_euler_flux_tensor}
\end{equation*}
Here, $\rho$ and $p$ represent density and pressure respectively, $u_i$ are the velocity components, and $\delta_{ij}$ is the Kronecker delta. The total energy $E \defeq e + \frac{q^2}{2}$ where $q$ is the specific kinetic energy and $e \defeq (\gamma-1)p/\rho$ for a calorically perfect gas with specific heat ratio $\gamma$.



\section{Numerical method}
\label{sec:num_method}

The split form DGSEM algorithm along with the subcell limiter of \citear{hennemannRamirezHindenlang2021} is used. The algorithm is described here in brief for the 1d system of equations
\begin{equation}
	\pder{\vect{\eulerphy{U}}}{t} + \pder{\vect{\eulerphy{F}}_1}{x_1} = \vect{0} \quad (t \geq 0,\ x_1 \in \Omega).
	\label{eq:1d_euler}
\end{equation}
For simplicity, the subscript `1' is dropped in this section. The domain $\Omega$ is partitioned into non-overlapping 1d elements, $\{\Omega_e\}_{e=0}^{N_e-1}$. Consider the restriction of \cref{eq:1d_euler} to a representative element $\Omega_e \defeq [a,b]$ which is being mapped to the reference element $\refelemone \defeq [0,1]$ using the linear mapping $\xi \xmapsto{\elemmapone} x$:
\begin{gather*}
	\elemmapone^{-1}(x) = \xi = \frac{x-a}{b-a} \quad (x \in \Omega_e \defeq [a,b]),
	\label{eq:1d_element_mapping}\\
	J \pder{\vect{\eulerref{U}}}{t} + \pder{\vect{\eulerref{F}}}{\xi} = \vect{0} \quad (t \geq 0,\ \xi \in \refelemone \defeq [0,1]),
	\label{eq:1d_euler_refelem}
\end{gather*}
where $\vect{\eulerref{U}}(\xi,t) \defeq \vect{\eulerphy{U}}(x,t)$, $\vect{\eulerref{F}}(\vect{\eulerref{U}}) \defeq \vect{\eulerphy{F}}(\vect{\eulerref{U}})$ are the conservative state and flux in reference space, and $J \defeq \sdd \elemmapone/\sdd \xi = (b-a)$ is the Jacobian of the mapping. $\vect{\eulerref{U}}$ and $\vect{\eulerref{F}}$ are approximated to lie in a linear polynomial space of $N$-th degree polynomials
\begin{equation*}
	\vect{\eulerref{U}}(\xi,t) \approx \sum_{i=0}^{N} \vect{\eulerref{U}}_i(t) \ell_i(\xi), \quad \eulerref{\vect{F}}(\xi,t) \approx \sum_{i=0}^{N} \eulerref{\vect{F}}(U_i(t)) \ell_i(\xi),
	\label{eq:1d_euler_state_flux_interpolation}
\end{equation*}
where $\{\ell_i(\xi)\}_{i=0}^{N}$ are the basis Lagrange functions constructed using the $(N+1)$ Gauss-Lobatto points as nodes. The error in this approximation is minimised using the Galerkin approach in a strong sense for the test functions $\{\ell_j\}_{j=0}^{N}$:
\begin{equation}
	\int_{0}^{1} \ell_j J \pder{\vect{\eulerref{U}}}{t} \sdi \xi + \int_{0}^{1} \ell_j \pder{\vect{\eulerref{F}}}{\xi} \sdi \xi + \left[\ell_j (\vect{\eulerref{F}}^*-\vect{\eulerref{F}}) \right]_{\xi=0}^{\xi=1} = \vect{0}\quad (j=0,1,\ldots,N),
	\label{eq:1d_euler_strong_form}
\end{equation}
where $\vect{\eulerref{F}}^*$ is the numerical flux function introduced at the element boundaries to ensure conservation. \cref{eq:1d_euler_strong_form} can be simplified by introducing the \newword{mass} and \newword{stiffness} matrices, $\linalgmat{M}$ and $\linalgmat{S}$
\begin{equation}
	\sum_{j=0}^{N} M_{ij} J \dot{\vect{\eulerref{U}}}_j + S_{ij} \vect{\eulerref{F}}_j + B_{ij} \left( \vect{\eulerref{F}}^*_j - \vect{\eulerref{F}}_j \right) = 0 \quad (i=0,1,\ldots,N),
	\label{eq:1d_euler_tensor_form1}
\end{equation}
where,
\begin{gather*}
	M_{ij} \defeq \int_{0}^{1} \ell_i \ell_j \sdi \xi \quad (i,j=0,1,\ldots,N),\\
	S_{ij} \defeq \int_{0}^{1} \ell_i \der{\ell_j}{\xi} \sdi \xi \quad (i,j=0,1,\ldots,N),
\end{gather*}
and $\linalgmat{B} \defeq \diag (-1,\ 0,\ \ldots,\ 0,\ 1)$ is the surface evaluation matrix of size $(N+1)$ as well. $\dot{\vect{\eulerref{U}}}_j$, $\vect{\eulerref{F}}_j$ and $\vect{\eulerref{F}}^*_j$, for $j=0,1,\ldots,N$, are the nodal values.

The DGSEM uses co-located quadrature to evaluate mass and stiffness matrices which simplifies the matrix element expressions, and also the strong form discretization given in \cref{eq:1d_euler_tensor_form1} as follows
\begin{gather}
	\linalgmat{M} = \diag(w_0,\ w_1,\ \ldots,\ w_N),
	\label{eq:dgsem_mass_matrix} \nonumber\\
	\linalgmat{S} = \linalgmat{M} \linalgmat{D}, \quad D_{ij} = \der{\ell_j}{\xi}(\xi_i) \quad (i=0,1,\ldots,N),
	\label{eq:dgsem_polynomial_derivative_matrix} \nonumber\\
	J \dot{\vect{\eulerref{U}}}_i + \sum_{j=0}^{N} \left[ D_{ij} \vect{\eulerref{F}}_j \right] - \frac{\delta_{0i}}{w_0} \left( \vect{\eulerref{F}}^*_0 - \vect{\eulerref{F}}_0 \right) + \frac{\delta_{Ni}}{w_N} \left( \vect{\eulerref{F}}^*_N - \vect{\eulerref{F}}_N \right) = 0 \quad (i=0,1,\ldots,N),
	\label{eq:1d_euler_tensor_form2}
\end{gather}
where $\{w_i\}_{i=0}^{N}$ are the weights of the GL quadrature and $\linalgmat{D}$ is the \newword{polynomial derivative} matrix. The co-located quadrature also gives a summation-by-parts property to $\linalgmat{S}$ which allows writing the volumetric term of \cref{eq:1d_euler_tensor_form2} as a FV flux difference \cite{fisherCarpenterNordstrom2013}:
\begin{equation*}
	\sum_{j=0}^{N} D_{ij} \vect{\eulerref{F}}_j = \frac{\hat{\vect{\eulerref{F}}}_{i+1} - \hat{\vect{\eulerref{F}}}_i}{w_i} \quad (i=0,1,\ldots,N),
	\label{eq:telescopic_form}
\end{equation*}
where $\{\hat{\vect{\eulerref{F}}}_i\}_{i=0}^{N+1}$ are the consistent volumetric fluxes defined on the \newword{flux points} $\{\hat{\xi}_i\}_{i=0}^{N+1}$. These flux points are related to the quadrature (or interpolation) points through the weights as follows.
\begin{equation*}
	\hat{\xi}_0 = \xi_0; \quad \hat{\xi}_{N+1} = \xi_N;\quad
	\hat{\xi}_{i+1} - \hat{\xi}_i = w_i\ \  (i=0,1,\ldots,N),
	\label{eq:flux_points}
\end{equation*}
While the relation between $\{\hat{\vect{\eulerref{F}}}_i\}_{i=0}^{N+1}$ and $\{\vect{\eulerref{F}}_j\}_{j=0}^{N}$ is given in \cite{fisherCarpenterNordstrom2013}, \citear{fisherCarpenter2013} suggested the \newword{split form} DGSEM, where the volumetric fluxes $\{\hat{\vect{\eulerref{F}}}_i\}_{i=0}^{N+1}$ are constructed independently using
\begin{equation*}
	\begin{gathered}
		\hat{\vect{\eulerref{F}}}_0 \approx \vect{\eulerref{F}}_0, \quad \hat{\vect{\eulerref{F}}}_{N+1} \approx \vect{\eulerref{F}}_N,\\
		\hat{\vect{\eulerref{F}}}_i \approx \sum_{j=i}^{N} \sum_{k=0}^{i-1} 2 w_k D_{kj} \vect{\eulerref{F}}^\#(\vect{\eulerref{U}}_k,\ \vect{\eulerref{U}}_j)\quad (i=1,2,\ldots,N),
	\end{gathered}
	\label{eq:split_form_subcell_flux}
\end{equation*}
where $\vect{\eulerref{F}}^\#$ is an entropy-satisfying \newword{volumetric two-point flux function}.

\printbibliography
\end{document}